


\documentclass[11pt,a4paper,roman]{moderncv}        % possible options include font size ('10pt', '11pt' and '12pt'), paper size ('a4paper', 'letterpaper', 'a5paper', 'legalpaper', 'executivepaper' and 'landscape') and font family ('sans' and 'roman')

% modern themes
\moderncvstyle{banking}                            % style options are 'casual' (default), 'classic', 'oldstyle' and 'banking'
\moderncvcolor{blue}                                % color options 'blue' (default), 'orange', 'green', 'red', 'purple', 'grey' and 'black'
%\renewcommand{\familydefault}{\sfdefault}         % to set the default font; use '\sfdefault' for the default sans serif font, '\rmdefault' for the default roman one, or any tex font name
\nopagenumbers{}                                  % uncomment to suppress automatic page numbering for CVs longer than one page

% character encoding
\usepackage[utf8]{inputenc}
\usepackage{fontawesome}
\usepackage{tabularx}
\usepackage{ragged2e}
% if you are not using xelatex ou lualatex, replace by the encoding you are using
%\usepackage{CJKutf8}                              % if you need to use CJK to typeset your resume in Chinese, Japanese or Korean

% adjust the page margins
\usepackage[scale=0.8]{geometry}
\usepackage{multicol}
%\setlength{\hintscolumnwidth}{3cm}                % if you want to change the width of the column with the dates
%\setlength{\makecvtitlenamewidth}{10cm}           % for the 'classic' style, if you want to force the width allocated to your name and avoid line breaks. be careful though, the length is normally calculated to avoid any overlap with your personal info; use this at your own typographical risks...

\usepackage{import}

% personal data
\name{Florin-Cătălin}{Ungureanu}

 %\title{Curriculum Vitae}                               % optional, remove / comment the line if not wanted
\address{___________________ }{}{}% optional, remove / comment the line if not wanted; the "postcode city" and and "country" arguments can be omitted or provided empty
% \phone[mobile]{909-839-3097}                   % optional, remove / comment the line if not wanted
% \phone[fixed]{01234 123456}                    % optional, remove / comment the line if not wanted
%\phone[fax]{+3~(456)~789~012}                      % optional, remove / comment the line if not wanted
% \email{xpan1@swarthmore.edu}                               % optional, remove / comment the line if not wanted
% \homepage{shawnpan.me}                         % optional, remove / comment the line if not wanted
% \extrainfo{}                 % optional, remove / comment the line if not wanted
%\photo[64pt][0.4pt]{picture}                       % optional, remove / comment the line if not wanted; '64pt' is the height the picture must be resized to, 0.4pt is the thickness of the frame around it (put it to 0pt for no frame) and 'picture' is the name of the picture file
%\quote{Some quote}                                 % optional, remove / comment the line if not wanted

% to show numerical labels in the bibliography (default is to show no labels); only useful if you make citations in your resume
%\makeatletter
%\renewcommand*{\bibliographyitemlabel}{\@biblabel{\arabic{enumiv}}}
%\makeatother
%\renewcommand*{\bibliographyitemlabel}{[\arabic{enumiv}]}% CONSIDER REPLACING THE ABOVE BY THIS

% bibliography with mutiple entries
%\usepackage{multibib}
%\newcites{book,misc}{{Books},{Others}}
  
\newcommand*{\customcventry}[7][.25em]{
  \begin{tabular}{@{}l} 
    {\bfseries #4}
  \end{tabular}
  \hfill% move it to the right
  \begin{tabular}{l@{}}
     {\bfseries #5}
  \end{tabular} \\
  \begin{tabular}{@{}l} 
    {\itshape #3}
  \end{tabular}
  \hfill% move it to the right
  \begin{tabular}{l@{}}
     {\itshape #2}
  \end{tabular}
  \ifx&#7&%
  \else{\\%
    \begin{minipage}{\maincolumnwidth}%
      \small#7%
    \end{minipage}}\fi%
  \par\addvspace{#1}}

\newcommand*{\customcvproject}[4][.25em]{
%   \vfill\noindent
  \begin{tabular}{@{}l} 
    {\bfseries #2}
  \end{tabular}
  \hfill% move it to the right
  \begin{tabular}{l@{}}
     {\itshape #3}
  \end{tabular}
  \ifx&#4&%
  \else{\\%
    \begin{minipage}{\maincolumnwidth}%
      \small#4%
    \end{minipage}}\fi%
  \par\addvspace{#1}}

\setlength{\tabcolsep}{12pt}

%----------------------------------------------------------------------------------
%            content
%----------------------------------------------------------------------------------
\begin{document}
%\begin{CJK*}{UTF8}{gbsn}                          % to typeset your resume in Chinese using CJK
%-----       resume       ---------------------------------------------------------
\makecvtitle
\vspace*{-23mm}

\begin{center}
\begin{tabular}{  c c c }
  \faEnvelopeO\enspace florin.g.ungureanu@gmail.com & \faGithub\enspace Florin-Catalin &  \faMobile\enspace hidden\\  
\end{tabular}
\end{center}

\section{EDUCATION}
{\customcventry{2016-2020}{Faculty of Automation and Computer Science}{Technical University of Cluj-Napoca}{Cluj-Napoca}{}{Bachelor of Technology in Computer Science}}

{\customcventry{2014-2016}{Mate-Info intensiv engleza   }{Colegiul National "Mihai Eminescu"}{Suceava}{}{}}

%\section{KEY SKILLS}

\section{PERSONAL PROJECTS}

{\customcvproject{Simple Remote Administration Tool}{ 2018 - Present}
  {\begin{itemize}
   \item Application aiming to gain information from a different machine
    \item One Client to many Servers architecture ( Client Server)
    \item JavaFX UI interface
    \item MVC,Object Pooling DP (Thread Pools) , Java Sockets
    \item Developed in NetBeans IDE
  \end{itemize}
  }
}

{\customcvproject{Multi-Threaded Customer Queue Simulator}{2018}
{\begin{itemize}
   \item Application aiming to analyze queuing based systems for determining and minimizing clients’ waiting time
   \item MVC DP
   \item Multi-Threaded
   \item Java Swing UI
   \item Developed in Eclipse IDE
\end{itemize}
}

{\customcvproject{Bank application }{2018}
{\begin{itemize}
  \item Application aiming to help user managing their accounts.Having two type of users with different responsibilities and two types of accounts.
  \item Design by contract DP
  \item JUnit testing 
  \item Java Swing UI
   \item Developed in Eclipse IDE
\end{itemize}
}

{\customcvproject{Warehouse Management System }{2018}
{\begin{itemize}
  \item Application aiming to manage commands using items stored in a Database
  \item Java Reflection, JDBC Driver
  \item Java Swing UI
\end{itemize}
}

{\customcvproject{Simple app with streams and Lambda Expression processing }{2018}
{\begin{itemize}
  \item Application aiming to parse a txt file with different activities 
  \item Developed as a console application using Java 8
\end{itemize}
}

{\customcvproject{Polynomial Calculator }{2018}
{\begin{itemize}
  \item Application aiming to parse a txt file with different activities 
  \item Developed as a console application using Java 8
\end{itemize}
}

}

\section{SKILL SET}
\begin{minipage}{\maincolumnwidth}%
	\small{
    	\begin{itemize}
          \item Winner, Tata Innovista, November 2016
          \item President of XYZ Club, Public Relations Manager of Swarthmore QWE Club
          \item Programming Languages: Python, C, C++, PHP, Java, HTML/CSS, Javascript, jQuery, NodeJS
          \item Fluent in Gibberish, conversational in Nonsense
		\end{itemize}}%
\end{minipage}%
      
}
% Publications from a BibTeX file without multibib
%  for numerical labels: \renewcommand{\bibliographyitemlabel}{\@biblabel{\arabic{enumiv}}}% CONSIDER MERGING WITH PREAMBLE PART
%  to redefine the heading string ("Publications"): \renewcommand{\refname}{Articles}
\nocite{*}
\bibliographystyle{plain}
\bibliography{publications}                        % 'publications' is the name of a BibTeX file

% Publications from a BibTeX file using the multibib package
%\section{Publications}
%\nocitebook{book1,book2}
%\bibliographystylebook{plain}
%\bibliographybook{publications}                   % 'publications' is the name of a BibTeX file
%\nocitemisc{misc1,misc2,misc3}
%\bibliographystylemisc{plain}
%\bibliographymisc{publications}                   % 'publications' is the name of a BibTeX file

%-----       letter       ---------------------------------------------------------

\end{document}


%% end of file `template.tex'.
